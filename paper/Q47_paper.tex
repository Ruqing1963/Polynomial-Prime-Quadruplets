\documentclass[11pt,a4paper]{article}
\usepackage[utf8]{inputenc}
\usepackage[T1]{fontenc}
\usepackage{amsmath,amssymb,amsthm}
\usepackage{mathtools}
\usepackage{graphicx}
\usepackage{booktabs}
\usepackage{array}
\usepackage{hyperref}
\usepackage{xcolor}
\usepackage{colortbl}
\usepackage{geometry}
\usepackage{enumitem}
\usepackage{float}
\geometry{margin=1in}

\hypersetup{
    colorlinks=true,
    linkcolor=blue,
    urlcolor=blue,
    citecolor=blue
}

\newtheorem{conjecture}{Conjecture}
\newtheorem{remark}{Remark}
\newtheorem{definition}{Definition}

\definecolor{highlight}{RGB}{255,249,196}
\definecolor{safegreen}{RGB}{46,125,50}

\title{\textbf{Prime Clustering in Polynomial $Q(n) = n^q - (n-1)^q$:}\\[0.5em]
\large Quadruplet Distribution and Riemann Zero Correlation\\[0.3em]
\normalsize An Empirical Study Based on 18 Million Verified Primes}

\author{Ruqing Chen\\[0.3em]
\small GUT Geoservice Inc., Montreal, Canada\\
\small \texttt{ruqing@hotmail.com}}

\date{January 2026}

\begin{document}

\maketitle

\begin{abstract}
We present a comprehensive computational study of prime-generating polynomials $Q(n) = n^q - (n-1)^q$ for seven prime exponents $q \in \{37, 41, 43, 47, 53, 61, 71\}$. Analyzing 18,356,706 verified primes in the range $n \in [1, 2 \times 10^9]$, we discover \textbf{15 quadruplet systems} for $Q_{47}$, compared to at most 2 for all other polynomials. A striking correlation ($r = 0.994$) emerges between quadruplet positions and Riemann zeta zeros under the scaling $n^{1/\alpha}$. We propose the \emph{Subgroup Interference Hypothesis}: the interference potential $I(q) = d(q-1) - 2$ determines clustering stability by minimizing the topological complexity of the sieve space. The derived effective modulus $q_{\mathrm{eff}} = 15.5 \pm 0.5 \approx q/3$ suggests triplet-induced dimensional reduction. $Q_{47}$ exemplifies rapid convergence to Bateman-Horn asymptotics due to minimal subgroup interference.

\medskip
\noindent\textbf{Keywords:} prime-generating polynomials, prime constellations, safe primes, Riemann zeta zeros, Bateman-Horn conjecture, subgroup lattice, effective modulus
\end{abstract}

%==============================================================================
\section{Introduction}
%==============================================================================

The Bateman-Horn conjecture \cite{bateman1962} provides a general framework for predicting prime densities in polynomial sequences. For a polynomial $f(x)$ of degree $d$, the asymptotic prime count is:
\begin{equation}
\pi_f(N) \sim \frac{C}{d} \cdot \frac{N}{\ln N}
\end{equation}
where $C$ is a product of local densities. In this paper, we investigate the family of polynomials:
\begin{equation}
\boxed{Q(n) = n^q - (n-1)^q}
\end{equation}
where $q$ is a prime exponent. We focus on the phenomenon of \emph{prime constellations}: consecutive integers $n, n+1, \ldots, n+k-1$ that all produce prime $Q$-values, forming doublets ($k=2$), triplets ($k=3$), and quadruplets ($k=4$).

Our central finding concerns $Q_{47}$, where $q = 47$ is a \textbf{safe prime}:
\begin{equation}
47 = 2 \times 23 + 1 \quad \text{(with 23 also prime)}
\end{equation}
This unique arithmetic property among our seven test polynomials correlates with remarkable structural stability, enabling the formation of 15 quadruplet systems up to $n = 2 \times 10^9$.

%==============================================================================
\section{Data and Methods}
%==============================================================================

\subsection{Dataset Construction}

We performed exhaustive primality testing on seven polynomial families with $q \in \{37, 41, 43, 47, 53, 61, 71\}$. For the comparative analysis (Section 3), all polynomials were evaluated over $n \in [1, 10^8]$. For $Q_{47}$, we extended the search to $n \in [1, 2 \times 10^9]$, yielding:
\begin{equation}
\text{Total verified primes: } \mathbf{18{,}356{,}706}
\end{equation}
Primality was confirmed using Miller-Rabin tests with deterministic witness sets \cite{miller1976}.

\subsection{Statistical Metrics}

\begin{definition}[Triplet Surplus]
The triplet surplus $S_3$ measures the percentage deviation of observed triplet count from Poisson expectation:
\begin{equation}
S_3 = \frac{O - E}{E} \times 100\%
\end{equation}
where $O$ is the observed count and $E$ is the expected count under random distribution.
\end{definition}

\begin{definition}[Interference Potential]
The interference potential $I(q)$ quantifies the complexity of the subgroup lattice $\mathcal{L}(\mathbb{Z}_q^*)$:
\begin{equation}
\boxed{I(q) = d(q-1) - 2}
\end{equation}
where $d(n)$ denotes the divisor function. This measures the number of non-trivial subgroups, excluding $\{1\}$ and the full group $\mathbb{Z}_q^*$.
\end{definition}

%==============================================================================
\section{Seven-Polynomial Comparison}
%==============================================================================

Table~\ref{tab:seven} summarizes the clustering statistics for all seven polynomials in the range $n \in [1, 10^8]$, with extended quadruplet census up to $n = 2 \times 10^9$.

\begin{table}[H]
\centering
\caption{Clustering Statistics with Extended Range Quadruplet Census}
\label{tab:seven}
\begin{tabular}{lcccccl}
\toprule
$Q_q$ & Safe? & $q-1$ factorization & $d(q-1)$ & $I(q)$ & $S_3$ & 4-lets ($n < 2 \times 10^9$) \\
\midrule
$Q_{37}$ & No & $2^2 \cdot 3^2$ & 9 & 7 & $+20.4\%$ & 2 (decay) \\
$Q_{41}$ & No & $2^3 \cdot 5$ & 8 & 6 & $+48.8\%$ & 0 \\
$Q_{43}$ & No & $2 \cdot 3 \cdot 7$ & 8 & 6 & $+15.3\%$ & 1 (decay) \\
\rowcolor{highlight}
$\mathbf{Q_{47}}$ & \textbf{Yes} & $\mathbf{2 \cdot 23}$ & \textbf{4} & $\mathbf{2}$ & $+35.0\%$ & \textbf{15 (growth)} \\
$Q_{53}$ & No & $2^2 \cdot 13$ & 6 & 4 & $+30.7\%$ & 0 \\
$Q_{61}$ & No & $2^2 \cdot 3 \cdot 5$ & 12 & 10 & $-16.7\%$ & 0 \\
$Q_{71}$ & No & $2 \cdot 5 \cdot 7$ & 8 & 6 & $+20.5\%$ & 0 \\
\bottomrule
\end{tabular}
\end{table}

\begin{figure}[H]
\centering
\includegraphics[width=0.85\textwidth]{fig2_seven_polynomials.png}
\caption{Triplet surplus $S_3$ distribution by residue class $p \bmod 6$ (left panel) and interference potential $I(q)$ versus triplet surplus (right panel). $Q_{47}$ (highlighted) shows moderate $S_3$ with minimal $I(q) = 2$.}
\label{fig:seven}
\end{figure}

\noindent\textbf{Key Observations:}
\begin{itemize}[leftmargin=2em]
\item $Q_{41}$ exhibits the highest triplet surplus ($+48.8\%$) but produces \emph{zero} quadruplets, indicating stochastic rather than structural clustering.
\item $Q_{61}$ with maximum $I(q) = 10$ shows negative surplus ($-16.7\%$), demonstrating subgroup-induced suppression.
\item $Q_{47}$ is the \textbf{only} polynomial showing quadruplet \emph{growth} with extended range.
\end{itemize}

\subsection{Subgroup Lattice Structure}

For safe prime $q = 47 = 2 \times 23 + 1$, the multiplicative group $\mathbb{Z}_{47}^*$ has a \emph{minimal} subgroup lattice---a simple linear chain:
\begin{equation}
\{1\} \subset H_2 \subset H_{23} \subset \mathbb{Z}_{47}^*
\end{equation}
In contrast, $Q_{41}$ (where $40 = 2^3 \times 5$) has a complex branching lattice with multiple maximal subgroups, creating high interference through path multiplicity.

\begin{figure}[H]
\centering
\includegraphics[width=0.9\textwidth]{fig4_lattice_comparison.png}
\caption{Subgroup lattice structure comparison. Left: $\mathbb{Z}_{47}^*$ exhibits a minimal linear chain $\{1\} \subset H_2 \subset H_{23} \subset \mathbb{Z}_{47}^*$ with $I(q) = 2$. Right: $\mathbb{Z}_{41}^*$ shows a complex branching network with multiple maximal subgroups and $I(q) = 6$.}
\label{fig:lattice}
\end{figure}

%==============================================================================
\section{$Q_{47}$ Quadruplet Census}
%==============================================================================

In the extended range $n \in [1, 2 \times 10^9]$, we identified \textbf{15 quadruplet systems} for $Q_{47}$. Table~\ref{tab:quads} provides the complete, verified coordinate list.

\begin{table}[H]
\centering
\caption{Complete Coordinate List of $Q_{47}$ Quadruplet Systems (Verified Data)}
\label{tab:quads}
\begin{tabular}{rrl}
\toprule
$k$ & Starting position $n$ & Range \\
\midrule
1 & $23{,}159{,}557$ & $0.02 \times 10^9$ \\
2 & $117{,}309{,}848$ & $0.12 \times 10^9$ \\
3 & $136{,}584{,}738$ & $0.14 \times 10^9$ \\
4 & $218{,}787{,}064$ & $0.22 \times 10^9$ \\
5 & $411{,}784{,}485$ & $0.41 \times 10^9$ \\
6 & $423{,}600{,}750$ & $0.42 \times 10^9$ \\
7 & $523{,}331{,}634$ & $0.52 \times 10^9$ \\
8 & $640{,}399{,}031$ & $0.64 \times 10^9$ \\
9 & $987{,}980{,}498$ & $0.99 \times 10^9$ \\
10 & $1{,}163{,}461{,}515$ & $1.16 \times 10^9$ \\
11 & $1{,}370{,}439{,}187$ & $1.37 \times 10^9$ \\
12 & $1{,}643{,}105{,}964$ & $1.64 \times 10^9$ \\
13 & $1{,}691{,}581{,}855$ & $1.69 \times 10^9$ \\
14 & $1{,}975{,}860{,}550$ & $1.98 \times 10^9$ \\
15 & $1{,}996{,}430{,}175$ & $2.00 \times 10^9$ \\
\bottomrule
\end{tabular}
\end{table}

\begin{figure}[H]
\centering
\includegraphics[width=0.75\textwidth]{fig3_quadruplet_positions.png}
\caption{Spatial distribution of $Q_{47}$ quadruplet systems across the search range $n \in [1, 2 \times 10^9]$. Notable clustering occurs near $n \approx 1.65 \times 10^9$ (positions 12--13) and $n \approx 1.99 \times 10^9$ (positions 14--15), suggesting resonance phenomena in the prime distribution.}
\label{fig:positions}
\end{figure}

%==============================================================================
\section{Correlation with Riemann Zeros}
%==============================================================================

Let $n_k$ denote the position of the $k$-th quadruplet, and let $\gamma_k$ denote the imaginary part of the $k$-th non-trivial zero of the Riemann zeta function. We performed regression analysis under various transformations.

\begin{table}[H]
\centering
\caption{Regression Analysis: Quadruplet Positions versus Riemann Zero Ordinates}
\label{tab:corr}
\begin{tabular}{lcc}
\toprule
Transformation & Correlation $r$ & $p$-value \\
\midrule
Scaled position comparison & $0.958$ & $< 0.001$ \\
\rowcolor{highlight}
\textbf{Optimal power transform} $n^{1/2.74}$ & $\mathbf{0.994}$ & $< 10^{-9}$ \\
Log-linear fit $\ln(n)$ vs $\gamma$ & $0.967$ & $4.4 \times 10^{-9}$ \\
\bottomrule
\end{tabular}
\end{table}

\noindent\textbf{Key Result:}
\begin{equation}
\boxed{n_k^{1/\alpha} \propto \gamma_k \quad \text{with } \alpha = 2.74, \; r = 0.994}
\end{equation}

\begin{figure}[H]
\centering
\includegraphics[width=0.9\textwidth]{fig1_riemann_correlation.png}
\caption{Linear regression analysis of quadruplet positions versus Riemann zeta zero ordinates. Left: Optimal power transform $n_k^{1/2.74}$ versus $\gamma_k$ yields $r = 0.994$. Right: Log-linear fit $\ln(n_k)$ versus $\gamma_k$ yields $r = 0.967$. Both methods confirm strong correlation between quadruplet positions and Riemann zeros.}
\label{fig:riemann}
\end{figure}

\subsection{Effective Modulus and the $q/3$ Conjecture}

A naive dimensional analysis suggests $\alpha \sim 1/\ln(q) = 1/\ln(47) \approx 1/3.85 \approx 0.26$. However, the observed exponent $\alpha_{\mathrm{obs}} = 1/2.74 \approx 0.365$ indicates a \emph{dimension compression effect}. We model this as $\alpha \sim 1/\ln(q_{\mathrm{eff}})$, yielding:
\begin{equation}
\boxed{q_{\mathrm{eff}} = e^{2.74} = 15.5 \pm 0.5}
\end{equation}

This value is remarkably close to:
\begin{equation}
\frac{q}{3} = \frac{47}{3} = 15.67
\end{equation}

This coincidence suggests a \textbf{dimensional reduction factor of 3}, corresponding to the triplet correlation structure inherent in quadruplet formation. The four consecutive primes are constrained by \emph{three independent gap conditions}, which may induce the observed $1/3$ reduction---analogous to Debye screening in plasma physics, where collective effects reduce the effective interaction range.

%==============================================================================
\section{Theoretical Framework}
%==============================================================================

\subsection{Subgroup Lattice Complexity}

We interpret $I(q) = d(q-1) - 2$ not merely as a subgroup count, but as an \emph{index of subgroup lattice complexity}. This measures the \textbf{interference cross-section}: how many distinct sieve paths can scatter prime candidates into composite residue classes.

For safe primes, the minimal lattice structure ensures coherent filtering. For $Q_{47}$ ($46 = 2 \times 23$), there is essentially one non-trivial sieve path. For $Q_{41}$ ($40 = 2^3 \times 5$), multiple maximal subgroups create a branching lattice with high interference.

\subsection{Heuristic Asymptotic Model}

\begin{conjecture}[Quadruplet Crystallization]
As $n \to \infty$, the relative density of $Q_{47}$ quadruplets diverges:
\begin{equation}
R(n) = \frac{P_{Q_{47}}(n)}{P_{\mathrm{rnd}}(n)} \sim (\ln n)^\Delta \to \infty
\end{equation}
where $\Delta > 0$ is a coherence correction arising from prime entanglement in the minimal lattice structure.
\end{conjecture}

\begin{remark}
A rigorous proof of this divergence would constitute progress toward the Hardy-Littlewood $k$-tuple conjecture for this polynomial family. Our 15 quadruplets up to $2 \times 10^9$, combined with $113\times$ doublet enhancement observed at $n \sim 10^{44}$, provide \textbf{strong numerical evidence} for structure persistence in the $Q_{47}$ waveguide.
\end{remark}

%==============================================================================
\section{Discussion}
%==============================================================================

\subsection{Bateman-Horn Convergence Rate}

Our results suggest that for safe prime polynomials like $Q_{47}$, the Bateman-Horn constant $C$ converges much faster to its asymptotic value due to minimal subgroup interference. The effective modulus $q_{\mathrm{eff}}$ can be interpreted as a \emph{correction factor} to the standard Bateman-Horn density prediction in the pre-asymptotic regime.

While $Q_{41}$ exhibits ``thermal noise'' (high $S_3$ but stochastic clustering), $Q_{47}$ exhibits ``coherent signal'' (moderate $S_3$ but structured), reaching asymptotic behavior at much smaller $n$.

\subsection{The $q/3$ Rule}

The coincidence $q_{\mathrm{eff}} \approx q/3$ (within experimental error) suggests a fundamental structural relation. In quadruplet formation, four consecutive primes are constrained by \emph{three independent gap conditions} (gaps 1, 2, 3). This triplet structure may induce the observed $1/3$ dimensional reduction, analogous to how correlated electron systems exhibit effective mass renormalization.

\subsection{Open Problems}

\begin{enumerate}[leftmargin=2em]
\item First-principles derivation of $q_{\mathrm{eff}} = q/3$ from Bateman-Horn constants $C_k$.
\item Rigorous proof of asymptotic divergence $R(n) \to \infty$.
\item Extension to other safe prime polynomials ($Q_5$, $Q_7$, $Q_{11}$, $Q_{23}$).
\item Verification of the $q/3$ rule across polynomial families.
\end{enumerate}

%==============================================================================
\section{Conclusion}
%==============================================================================

Through exhaustive computation on $18{,}356{,}706$ verified primes, we establish four principal results:

\begin{enumerate}[leftmargin=2em]
\item \textbf{Empirical Dominance:} $Q_{47}$ produces 15 quadruplets versus at most 2 for all other polynomials tested (Table~\ref{tab:seven}).

\item \textbf{Mechanistic Insight:} Minimal subgroup lattice complexity ($I(q) = 2$) enables coherent prime clustering, while complex lattices ($I(q) \geq 6$) produce stochastic or suppressed behavior.

\item \textbf{Scaling Law:} Quadruplet positions correlate with Riemann zeros ($r = 0.994$) under the transformation $n^{1/2.74}$. The effective modulus $q_{\mathrm{eff}} = 15.5 \pm 0.5 \approx q/3$ quantifies coherence-induced screening.

\item \textbf{Bateman-Horn Interpretation:} $Q_{47}$ exemplifies rapid convergence to asymptotic behavior, making it an ideal ``arithmetic laboratory'' for studying prime constellation dynamics.
\end{enumerate}

These findings support the conjecture that safe prime polynomials possess unique asymptotic coherence properties linked to Riemann zeta zeros. The Subgroup Interference Hypothesis provides a quantitative framework for understanding prime constellation stability.

%==============================================================================
\section*{Data Availability}
%==============================================================================

Complete datasets (18,356,706 $Q_{47}$ primes, all 15 quadruplet coordinates with cryptographic verification, seven-polynomial statistics) and analysis source code are available at:

\begin{center}
\url{https://github.com/Ruqing1963/Polynomial-Prime-Quadruplets}
\end{center}

%==============================================================================
\begin{thebibliography}{9}
%==============================================================================

\bibitem{bateman1962}
Bateman, P.T. \& Horn, R.A. (1962).
A heuristic asymptotic formula concerning the distribution of prime numbers.
\emph{Mathematics of Computation}, 16(79), 363--367.

\bibitem{miller1976}
Miller, G.L. (1976).
Riemann's hypothesis and tests for primality.
\emph{Journal of Computer and System Sciences}, 13(3), 300--317.

\bibitem{hardy1923}
Hardy, G.H. \& Littlewood, J.E. (1923).
Some problems of `Partitio Numerorum' III: On the expression of a number as a sum of primes.
\emph{Acta Mathematica}, 44, 1--70.

\bibitem{montgomery1973}
Montgomery, H.L. (1973).
The pair correlation of zeros of the zeta function.
\emph{Proceedings of Symposia in Pure Mathematics}, 24, 181--193.

\bibitem{odlyzko1987}
Odlyzko, A.M. (1987).
On the distribution of spacings between zeros of the zeta function.
\emph{Mathematics of Computation}, 48(177), 273--308.

\bibitem{selberg1947}
Selberg, A. (1947).
An elementary proof of the prime-number theorem.
\emph{Annals of Mathematics}, 50(2), 305--313.

\bibitem{titchmarsh1986}
Titchmarsh, E.C. (1986).
\emph{The Theory of the Riemann Zeta-Function}.
Oxford University Press, 2nd edition.

\bibitem{ribenboim2004}
Ribenboim, P. (2004).
\emph{The Little Book of Bigger Primes}.
Springer, 2nd edition.

\end{thebibliography}

\end{document}
